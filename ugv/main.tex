\chapter{UGV}
This chapter describes the setup of an Unmanned Ground Vehicle (UGV) and a few 
experiments that can be performed on it with the Vaman board.
\section{Components Table}
The components of the UGV kit are listed in \autoref{Tab:components}.
\begin{table}[!ht]
\centering
	\input{ugv/tables/components}
	\caption{components table of toycar}
	\label{Tab:components}
\end{table}
\begin{figure}[!ht]
\centering
\includegraphics[width=0.5\columnwidth]{ugv/figs/motor.png}
\caption{DC motors}
\label{fig:motor}
\end{figure}

\begin{figure}[!ht]
\centering
\includegraphics[width=0.3\columnwidth]{ugv/figs/driver.jpg}
\caption{L293 motor driver}
\label{fig:l293}
\end{figure}

\begin{figure}[!ht]
\centering
\includegraphics[width=0.5\columnwidth]{ugv/figs/base.png}
\caption{UGV frame/chassis}
\label{fig:frame}
\end{figure}

\begin{figure}[!ht]
\centering
\includegraphics[width=0.5\columnwidth]{ugv/figs/battery.png}
\caption{Batteries for powering various equipments}
\label{fig:battery}
\end{figure}

\section{Assembling the UGV kit}

Assemble the Chassis using the provided nuts/screws, Wheels, and parts. 
 
 \begin{figure}[!ht]
\centering
\includegraphics[width=0.3\columnwidth]{ugv/figs/2.png}
\centering
\caption{screws connecting }
\end{figure}
 
 \begin{figure}[!ht]
\centering
\includegraphics[width=0.3\columnwidth]{ugv/figs/3.png}
\caption{Dc motors connecting}
\end{figure}
 \begin{figure}[!ht]
\centering
\includegraphics[width=0.3\columnwidth]{ugv/figs/6.jpg}
\caption{wheels connections }
\end{figure}
\section{Circuit Connections}

\begin{table}[!ht]
\centering
	\input{ugv/tables/motor}
	\caption{DC motor connection with L293 Motor Driver }
	\label{Tab:dcmotor}
\end{table}
%\item  Make the Circuit Connections as per the  \autoref{Tab:dcmotor}

\begin{table}[!ht]
\centering
	\input{ugv/tables/table3}
	\caption{ vaman Connections}
	\label{Tab:connections3}
\end{table}
%\item Make the Circuit Connections as per the \autoref{Tab:connections3}.
\begin{enumerate}[label=\thesection.\arabic*.,ref=\thesection.\theenumi]

\item Connect the Arduino-UART pins to the Vaman ESP32 pins according to \autoref{tab:arduino-uart}.
\item For Bluetooth toycar connect the circuits connection as per \autoref{Tab:dcmotor} and  \autoref{Tab:connections3}.
\begin{table}[!ht]
\centering
	\input{ugv/tables/table1}
	\caption{connection with vaman board }
	\label{Tab:connections}
\end{table}
\begin{table}[!ht]
%\item Make the Motor Driver Connections as per the \autoref{Tab:connections}
\centering
	\input{ugv/tables/table2}
	\caption{vaman connection with L293 Motor Driver}
	\label{Tab:connections2}
\end{table}
\item For \textbf{Integrated Bluetooth toycar} connect the circuits connection
as per \autoref{Tab:dcmotor} , \autoref{Tab:connections} and
\autoref{Tab:connections2}.

%\item Connect the Arduino-UART pins to the Vaman ESP32 pins according to \autoref{tab:arduino-uart}.
\begin{figure}[!ht]
\centering
\includegraphics[width=0.3\columnwidth]{ugv/figs/8.jpg}
\caption{After all connections}
\end{figure}
\end{enumerate}

\section{Code Execution For Bluetooth Toycar}
\begin{enumerate}[label=\thesection.\arabic*.,ref=\thesection.\theenumi]
\item Now,Execute the following code
\begin{lstlisting}
vaman/toycar/codes/bluetooth_toycar/src
\end{lstlisting}

\item Build the ESP32 firmware
\begin{lstlisting}
cd  vaman/toycar/codes/bluetooth_toycar
pio run
\end{lstlisting} 

\item Flash ESP32 firmware (connect Arduino-UART)
\begin{lstlisting}
pio run -t upload
\end{lstlisting} 
\item Install the \textbf{Dabble app} on the Mobile from the \textbf{Playstore}. Connect it to the \textbf{ESP32} on the Vaman Board using \textbf{Bluetooth}. Change the controls to \textbf{Joystick mode} as shown in \autoref{fig:ble_app}to navigate the UGV.\\
\end{enumerate}

\section{Code Execution for Integrated Bluetooth Toycar}
\begin{enumerate}[label=\thesection.\arabic*.,ref=\thesection.\theenumi]

\item Now,Execute the following code 

\begin{lstlisting}
vaman/toycar/codes/bluetooth_toycar
\end{lstlisting}

\item Build the ESP32 firmware
\begin{lstlisting}
cd esp32_pwmctrl
pio run
\end{lstlisting} 

\item Flash ESP32 firmware (connect Arduino-UART)
\begin{lstlisting}
pio run -t nobuild -t upload
\end{lstlisting} 

\item If using termux, send .pio/build/esp32doit-devkit-v1/firmware.bin to PC using
\begin{lstlisting}
scp .pio/build/esp32doit-devkit-v1/firmware.bin Username@IPAddress:
\end{lstlisting} 

\item  Modify line 140 of config.mk to setup path to pygmy-sdk and then Build m4 firmware using
\begin{lstlisting}
cd m4_pwmctrl/GCC_Project
make
\end{lstlisting}

\item If using termux, send output/m4{\_}pwmctrl.bin to PC using
\begin{lstlisting}
scp output/m4_pwmctrl.bin username@IPaddress:
\end{lstlisting} 

\item Build fpga source (.bin file)
\begin{lstlisting}
cd fpga_pwmctrl/rtl
ql_symbiflow -compile -d ql-eos-s3 -P pu64 -v *.v -t AL4S3B_FPGA_Top -p quickfeather.pcf -dump jlink binary 
\end{lstlisting} 

\item If using termux, send AL4S3B{\_}FPGA{\_}Top.bin to PC using
\begin{lstlisting}
scp AL4S3B_FPGA_Top.bin username@IPaddress:
\end{lstlisting} 

\item Connect usb cable to vaman board and Flash eos s3 soc, using
\begin{lstlisting}
sudo python3 <Type path to tiny fpga programmer application> --port /dev/ttyACM0  --appfpga AL4S3B_FPGA_Top.bin --m4app m4_pwmctrl.bin --mode m4-fpga --reset
\end{lstlisting} 

\item Install the \textbf{Dabble app} on the Mobile from the \textbf{Playstore}.
Connect it to the \textbf{ESP32} on the Vaman Board using \textbf{Bluetooth}.
Change the controls to \textbf{Joystick mode} as shown in
\autoref{fig:ble_app} to navigate the UGV.
\end{enumerate}
\subsection{Working}
On the hardware level there are three key points: SPI,Wishbone Interfacing and Address Mapping. Vaman Board, we have an EOS S3 and ESP32. The Communication between these two happens via Serial Peripheral Interface(SPI).\\
\vspace{0.25cm}

\begin{figure}[!ht]
\centering
\includegraphics[width=0.5\columnwidth]{ugv/figs/block3}
\centering
\caption{EOS S3 Architecture }
\end{figure}

\begin{figure}[!ht]
\centering
\includegraphics[width=0.5\columnwidth]{ugv/figs/block4}
\centering
\caption{Wishbone Slave Interface  }
\end{figure}
\begin{figure}[!ht]
\centering
\includegraphics[width=0.5\columnwidth]{ugv/figs/block5}
\centering
\caption{Hardware Block level Architecture }
\end{figure}
