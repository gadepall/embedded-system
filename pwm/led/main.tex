\section{Onboard LED}
This section illustrates the use of SPI Communication between the Vaman-ESP32
and the Vaman-Pygmy in adjusting the brightness of an LED onboard the 
Vaman-Pygmy.

\subsection{Components}
The components required are listed in \autoref{tab:pwm-led-components}.
\begin{table}[!ht]
    \centering
    \input{pwm/led/tables/components.tex}
    \caption{Components Required for Controlling the Onboard LED via SPI.}
    \label{tab:pwm-led-components}
\end{table}

\subsection{Connections}
The connections on the Vaman board are listed in
\autoref{tab:pwm-led-connections}.
\begin{table}[!ht]
    \centering
    \input{pwm/led/tables/connections.tex}
    \caption{Connections to establish SPI between Vaman-ESP32 and Vaman-Pygmy.}
    \label{tab:pwm-led-connections}
\end{table}

\subsection{Building}
\begin{enumerate}
    \item Build the PlatformIO project at
    \begin{lstlisting}
pwm/led/codes/esp32
    \end{lstlisting}
    \item Flash the project .bin file using USB-UART connected to the 
    Vaman-ESP32, via PlatformIO or ArduinoDroid.
    \item Build the FPGA project .bin file by entering the following commands at
    a terminal window.
    \begin{lstlisting}
# The following variable can be sourced from .shellrc or .venv/bin/activate
export QORC_SDK_PATH=/path/to/pygmy-sdk
cd pwm/led/codes/fpga
make
    \end{lstlisting}
    \item On building successfully, the file
    \begin{lstlisting}
pwm/led/codes/fpga/rtl/AL4S3B_FPGA_Top.bin
    \end{lstlisting}
    is generated.
    \item Edit the variable PROJ\_ROOT in
    \begin{lstlisting}
pwm/led/codes/m4/GCC_Project/config.mk
    \end{lstlisting}
    to point to the location of the pygmy-sdk folder on your system.
    \item Build the M4 project .bin file by entering the following commands at a
    terminal window.
    \begin{lstlisting}
cd pwm/led/codes/m4/GCC_Project
make -j4
    \end{lstlisting}
    \item Flash both FPGA and M4 .bin files to the Vaman-Pygmy by resetting it
    to bootloader mode and entering the following command at a terminal window.
    \begin{lstlisting}
python3 /path/to/tinyfpga-programmer-gui.py --port /dev/ttyACMxx --mode m4-fpga --m4app /path/to/m4.bin --appfpga /path/to/AL4S3B_FPGA_Top.bin --reset
    \end{lstlisting}
    where /dev/ttyACMxx is the port at which the Vaman board is available. This
    can be obtained by inspecting the output of the following command (requires
    root/sudo privileges).
    \begin{lstlisting}
dmesg -w
    \end{lstlisting}
\end{enumerate}

\subsection{Demonstration}
\begin{enumerate}[resume]
    \item Find the IP address of the Vaman-ESP32 by inspecting the output of the
    serial terminal, or by typing at a terminal window
    \begin{lstlisting}
ifconfig
nmap -sn xx.yy.zz.0/24
    \end{lstlisting}
    where xx.yy.zz represents the first three octets of the IP address of your
    device on the WiFi network interface, found using \texttt{ifconfig}.
    \item Then, go to the site 
    \begin{lstlisting}
http://<VAMAN-IP>/pwm 
    \end{lstlisting}
    and enter the PWM value, which is an integer between 0 and 255.
    \item On entering the value, the brightness of the green LED will change 
    according to the PWM value.
\end{enumerate}

\subsection{Exercises}
\begin{enumerate}[resume]
    \item Add code to gradually increase or decrease brightness of the onboard
    LED.
    \item Avoid using the M4 platform entirely by having the ESP32 write to the
    appropriate memory address of the PWM using SPI directly.
\end{enumerate}