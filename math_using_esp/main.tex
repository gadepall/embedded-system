\chapter{Maths computing using ESP32}
This chapter will demonstrate how to solve math problems using esp.
\section{Math Computing using ESP32}

\begin{enumerate}[label=\thesection.\arabic*.,ref=\thesection.\theenumi]

	\item Show that the points A $\myvec{1\\-2\\-8}$, B $\myvec{5\\0\\-2}$ and C $\myvec{11\\3\\7}$ are collinear, and find the ratio in which B divides AC.\\


\solution \\The input parameters for this problem are available in Table \ref{Table-1}
\begin{table}[ht!]
\input{./math_using_esp/tables/table.tex}
\caption{}
\label{Table-1}	

\end{table}

		Points $\vec{A}$, $\vec{B}$ and $\vec{C}$ are on a line if
    \begin{align}
        \textrm{rank}\myvec{\vec{A} & \vec{B} & \vec{C}} < 3
        \label{eq:chapters/12/10/5/8rank-collinear}
    \end{align}
    Substituting, we must find the rank of
    \begin{align}
        \vec{M} = \myvec{1&5&11\\-2&0&3\\-8&-2&7}
    \end{align}
    Using row reduction methods to bring $\vec{M}$ into row-reduced echelon form,
    \begin{align}
        \myvec{1&5&11\\-2&0&3\\-8&-2&7}&\xleftrightarrow[]{R_2\rightarrow R_2+2R_1}
        \myvec{1&5&11\\0&10&25\\-8&-2&7} \\
                &\xleftrightarrow[]{R_3\rightarrow R_3+8R_1}\myvec{1&5&11\\0&10&25\\0&38&95} \\
                &\xleftrightarrow[]{R_3\rightarrow R_3-\frac{19}{5}R_2}\myvec{1&5&11\\0&10&25\\0&0&0}
                \label{eq:chapters/12/10/5/8row-red}
    \end{align}
    Clearly, the rank of $\vec{M}$ is 2, and hence the given points are 
    collinear. 
    Fig. \ref{fig:Fig1}  verifies that the three points are indeed 
    collinear as claimed.\\
	Let $\vec{B}$ divide $\vec{AC}$ in k:1 then,
	\begin{align}
		\frac{k\vec{C}+\vec{A}}{k+1} = \vec{B}
	\end{align}
		\begin{align}
			\implies k\vec{C}+\vec{A}=\vec{B}\brak{k+1}
			\implies k\brak{\vec{C}-\vec{B}}=\brak{\vec{B}-\vec{A}}
		\end{align}
			Multiplying with $\brak{\vec{C}-\vec{B}}^{\top}$ on both sides,\\
	
		\begin{align*}
			 k\brak{\vec{C}-\vec{B}}\brak{\vec{C}-\vec{B}}^{\top}=\brak{\vec{B}-\vec{A}}{\vec{C}-\vec{B}}^{\top}
		\end{align*}
			The value of k is as follows,
			\begin{align}
			k &=
			\frac{\brak{\vec{B}-\vec{A}}\brak{\vec{C}-\vec{B}}^{\top}}{\norm{\vec{C-B}}^2}
			\end{align}
			where,
			\begin{align}
				\brak{\vec{B-A}} &=
				\brak{\myvec{5\\0\\-2}-\myvec{1\\-2\\-8}} =
				\myvec{4\\2\\6}
			\end{align}
			
			\begin{align}
				\brak{\vec{C-B}} &=
				\brak{\myvec{11\\3\\7}-\myvec{5\\0\\-2}} =
				\myvec{6\\3\\9}
			\end{align}
			\begin{align*}
				\brak{\vec{C}-\vec{B}}^{\top} &=
				\myvec{6 & 3 & 9}
			\end{align*}
			Substituting the values the value of $k$ is $2/3$.
			\\    Hence, $\vec{B}$ divides $\vec{AC}$ in the ratio $2:3$.
	\begin{figure}[!h]
		\begin{center}
			\includegraphics[width=\columnwidth]{./math_using_esp/figs/line_3d.png}
		\end{center}
		\caption{}
		\label{fig:Fig1}
	\end{figure}
\end{enumerate}

\begin{enumerate}[label=\thesection.\arabic*.,ref=\thesection.\theenumi]	
\subsection{C pointer code}
\item The following is the C code for the above question using pointers
\begin{lstlisting}
vaman/vaman-esp/math_using_esp/codes/cprog/main.c
\end{lstlisting}
\end{enumerate}

\begin{enumerate}[label=\thesection.\arabic*.,ref=\thesection.\theenumi]
\section{Code execution for Math Computing using esp32}
\raggedright
\item Now, Execute the following code
\begin{lstlisting}
math_using_esp/codes/esp_code/src/main.cpp
\end{lstlisting}
\item Build the ESP32 firmware
\begin{lstlisting}
cd  math_using_esp/codes/esp_code
pio run
\end{lstlisting} 
\item Flash ESP32 firmware ( using Arduino  )
\begin{lstlisting}
pio run -t upload
\end{lstlisting} 
\end{enumerate}

\subsection{Displaying the output  on website}
\begin{enumerate}[label=\thesection.\arabic*.,ref=\thesection.\theenumi]
\item The output of the code i.e; if the points are collinear or not will be displayed in the webserver.
\item After flashing the code to vaman-ESP, the board will be connected to the wifi credentials provided.
\item Now connect the same WiFi credentials to the mobile phone for accessing the IP address, which can be accessed by 
\begin{lstlisting}
ifconfig
nmap -sn 192.168.x.x/24
\end{lstlisting}
\item Change the IP address in the second command accordingly with the IP address provided by first command.
\item By the above commands the IP address of vaman-ESP will be diplayed.
\item Now the vaman-ESP will be hosting a webserver
\item In order to access the webserver type the IP address of the vaman-ESP in the web browser.
\item In the website loaded by the IP address of vaman-ESP the output is displayed as shown in Fig. \ref{fig:results of collinear}
\begin{figure}[H]
\centering
\includegraphics[scale =0.3]{./math_using_esp/figs/1.jpg}
\caption{Website}
\label{fig:results of collinear}
\end{figure}

\begin{figure}[H]
\centering
\includegraphics[width=\columnwidth]{./math_using_esp/figs/2.jpg}
\caption{Website result}
\label{fig:results of collinear equations }
\end{figure}

\end{enumerate}


