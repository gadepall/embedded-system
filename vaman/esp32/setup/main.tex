Here we show how to program the ESP32 on the Vaman using the Arduino framework.
%\subsubsection{Connections}
\begin{enumerate}[label=\arabic*.,ref=\theenumi]
\item Make sure that Vaman board does not power any devices.  
\item Make connections as shown in \autoref{tab:arduino-uart}.
	%and \autoref{fig:vaman/uart/1}. 
\item The Vaman pin diagram is available in \autoref{fig:vaman-pin_sheet}.
	\iffalse
\begin{figure}[!ht]
\centering
    \resizebox{\columnwidth}{!}{%
\input{vaman-esp32/arduino-uart/figs/circuit.tex}
	}
\caption{Circuit Connections}
\label{fig:vaman/uart/1}
\end{figure}
\fi
\begin{figure*}
\centering
\includegraphics[width=\columnwidth]{vaman-esp32/lcd/figs/pin_sheet.png}
\caption{Vaman pins.  Right side reperesents ESP32 and left side M4-FPGA.}
\label{fig:vaman-pin_sheet}
\end{figure*}
\begin{table}[!ht]
\input{vaman-esp32/arduino-uart/tables/rpi-vaman-uart.tex}
\caption{}
\label{tab:arduino-uart}
\end{table}
%
\item Connect the Arduino-UART to raspberry pi through USB.  
	\iffalse
\item On the rpi
\begin{lstlisting}
dmesg | tail
lsusb
\end{lstlisting}
you should see the USB-UART connector detected. 
\item Connect the Arduino-UART pins to the Vaman ESP32 pins according to Table 
		\ref{tab:vaman/esp32/setup/vaman-uart}
	\begin{table}[!h]
		\input{vaman/esp32/setup/tables/vaman-uart.tex}
		\caption{}
		\label{tab:vaman/esp32/setup/vaman-uart}
	\end{table}
	\fi

\item Connect the Vaman-ESP pins to the seven segment display  according to Table 
		\ref{tab:vaman/esp32/setup/esp-display}
	\begin{table}[!h]
		\input{vaman/esp32/setup/tables/esp-display.tex}
		\caption{}
		\label{tab:vaman/esp32/setup/esp-display}
	\end{table}
	\iffalse
	The GPIO pins are listed in Table 
		\ref{tab:vaman/esp32/setup/esp-pins}
	\begin{table}[!h]
		\input{vaman/esp32/setup/tables/esp-pins.tex}
		\caption{}
		\label{tab:vaman/esp32/setup/esp-pins}
	\end{table}
	Note that these pins can be used for several functions, refer to the ESP32 datasheet
		for details.  
		The Vaman pin diagram is available in 
\figref{fig:vaman/fpga/setup/pin_sheet}
		Fig. \ref{fig:vaman/esp32/setup/lc}
	\begin{figure}[!h]
		\includegraphics[width=\columnwidth]{vaman/esp32/setup/figs/lc.jpg}
		\caption{}
		\label{fig:vaman/esp32/setup/lc}
	\end{figure}
	\fi
\item On termux on your phone, 
\begin{lstlisting}
cd vaman/esp32/codes/ide/blink
pio run
\end{lstlisting}
\item Transfer the ini and bin files to the rpi 
\begin{lstlisting}
scp platformio.ini pi@192.168.50.252:./hi/platformio.ini

scp .pio/build/esp32doit-devkit-v1/firmware.bin pi@192.168.50.252:./hi/.pio/build/esp32doit-devkit-v1/firmware.bin
\end{lstlisting}
\item On rpi,
\begin{lstlisting}
cd /home/pi/hi
pio run -t nobuild -t upload
\end{lstlisting}
\item On your phone, open 
\begin{lstlisting}
src/main.cpp 
\end{lstlisting}
and change the delay to 
\begin{lstlisting}
delay(100);
\end{lstlisting}
and execute the code by following the steps above.
\iffalse
\end{enumerate}
\subsubsection{ESP IDF}
\renewcommand{\theequation}{\theenumi}
\renewcommand{\thefigure}{\theenumi}
\begin{enumerate}[label=\thesection.\arabic*.,ref=\thesection.\theenumi]
\numberwithin{equation}{enumi}
\numberwithin{figure}{enumi}
\numberwithin{table}{enumi}
\item Earlier, we were using the arduino framework, where the programming language was arduino.  In the following directory, the same functionality is achieved through a C program.
\begin{lstlisting}
cd vaman/esp32/codes/idf/blink
pio run
\end{lstlisting}
\item The flashing process remains the same.
\end{enumerate}


\subsubsection{OTA}
\renewcommand{\theequation}{\theenumi}
\renewcommand{\thefigure}{\theenumi}
\begin{enumerate}[label=\thesection.\arabic*.,ref=\thesection.\theenumi]
\numberwithin{equation}{enumi}
\numberwithin{figure}{enumi}
\numberwithin{table}{enumi}
\fi
\item Flash the following code. 
\begin{lstlisting}
vaman/esp32/codes/ide/ota/setup
\end{lstlisting}
		after entering your wifi username and password (in quotes below)
\begin{lstlisting}
#define STASSID "..." // Add your network credentials
#define STAPSK  "..."
\end{lstlisting}
in src/main.cpp file
\item You should be able to find the ip address of your vaman-esp using 
\begin{lstlisting}
ifconfig
nmap -sn 192.168.231.1/24
\end{lstlisting}
where your computer's ip address is the output of ifconfig and given by 192.168.231.x
\item Assuming that the username is gvv and password is abcd, flash the following code wirelessly
\begin{lstlisting}
vaman/esp32/codes/ide/ota/blink
\end{lstlisting}
through 
\begin{lstlisting}
pio run 
pio run -t nobuild -t upload --upload-port 192.168.231.245
\end{lstlisting}
where you may replace the above ip address with the ip address of your vaman-esp.
\item Connect pin 2 to an LED  to see it blinking.
	\iffalse
\end{enumerate}
\subsubsection{Onboard LED}
\renewcommand{\theequation}{\theenumi}
\renewcommand{\thefigure}{\theenumi}
\begin{enumerate}[label=\thesection.\arabic*.,ref=\thesection.\theenumi]
\numberwithin{equation}{enumi}
\numberwithin{figure}{enumi}
\numberwithin{table}{enumi}
\fi
\item Connect the pins between Vaman-ESP32 and Vaman-PYGMY as per Table \ref{tab:vaman/esp32/setup/onboard}
\begin{table}[h]
\centering
\input{./vaman/esp32/setup/tables/onboard.tex}
\caption{}
\label{tab:vaman/esp32/setup/onboard}
\end{table}
\item Flash the following code OTA
\begin{lstlisting}
vaman/esp32/codes/ide/ota/blinkt
\end{lstlisting}
You should see the onboard LEDs blinking.
\item Change the blink duration to 100 ms.
\end{enumerate}

