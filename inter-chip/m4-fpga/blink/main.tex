\subsection{Onboard LED}
This subsection illustrates the interface between the ARM Cortex-M4 subsystem
and the eFPGA subsystem onboard the Vaman-Pygmy through a blink program, where
the GPIO pins corresponding to the onboard LEDs are exposed by the FPGA
platform.

\subsubsection{Components}
The components required are listed in
\autoref{tab:esp32-m4-fpga-blink-components}.
\begin{table}[!ht]
    \centering
    \input{inter-chip/m4-fpga/blink/tables/components.tex}
    \caption{Components Required for Controlling the Onboard LED.}
    \label{tab:esp32-m4-fpga-blink-components}
\end{table}

\subsubsection{Building}
\begin{enumerate}
    \item Build the FPGA project .bin file by entering the following commands at
    a terminal window.
    \begin{lstlisting}
# The following variable can be sourced from .shellrc or .venv/bin/activate
export QORC_SDK_PATH=/path/to/pygmy-sdk
cd inter-chip/m4-fpga/blink/codes/fpga
make
    \end{lstlisting}
    \item On building successfully, the .bin file is generated at
    \begin{lstlisting}
inter-chip/m4-fpga/blink/codes/fpga/rtl/AL4S3B_FPGA_Top.bin
    \end{lstlisting}
    \item Edit the variable PROJ\_ROOT in the following file to point to the
    location of the pygmy-sdk folder on your system.
    \begin{lstlisting}
inter-chip/m4-fpga/blink/codes/m4/GCC_Project/config.mk
    \end{lstlisting}
    \item Build the M4 project .bin file by entering the following commands at a
    terminal window.
    \begin{lstlisting}
cd inter-chip/m4-fpga/blink/codes/m4/GCC_Project
make -j4
    \end{lstlisting}
    \item On building successfully, the .bin file is generated at
    \begin{lstlisting}
inter-chip/m4-fpga/blink/codes/m4/GCC_Project/output/bin/m4.bin
    \end{lstlisting}
    \item Flash both FPGA and M4 .bin files to the Vaman-Pygmy by resetting it
    to bootloader mode and entering the following command at a terminal window.
    \begin{lstlisting}
python3 /path/to/tinyfpga-programmer-gui.py --port /dev/ttyACMxx --mode m4-fpga --m4app /path/to/m4.bin --appfpga /path/to/AL4S3B_FPGA_Top.bin --reset
    \end{lstlisting}
    where /dev/ttyACMxx is the port at which the Vaman board is available. This
    can be obtained by inspecting the output of the following command (requires
    root/sudo privileges).
    \begin{lstlisting}
dmesg -w
    \end{lstlisting}
\end{enumerate}

\subsubsection{Demonstration}
All three LEDs onboard the Vaman will toggle every one second when the Vaman
boots normally.

\subsubsection{Exercises}
\begin{enumerate}[resume]
    \item Find out the addresses corresponding to the GPIO module by inspecting
    the code in pygmy-sdk, as well as in this folder.
    \item Create code to blink each LED in turn.
\end{enumerate}