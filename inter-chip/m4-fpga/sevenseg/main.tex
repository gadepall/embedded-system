\subsection{Seven-Segment Display}
This subsection demonstrates the interface between the M4 subsystem and eFPGA
subsystem on the Vaman-Pygmy by controlling a seven-segment display, where the
GPIO interface is exposed to the M4 by the FPGA platform.

\subsubsection{Components}
The components required are listed in \autoref{tab:m4-fpga-sevenseg-components}.
\begin{table}[!ht]
    \centering
    \input{inter-chip/m4-fpga/sevenseg/tables/components.tex}
    \caption{Components Required for Controlling the Onboard LED.}
    \label{tab:m4-fpga-sevenseg-components}
\end{table}

\subsubsection{Connections}
The connections between the seven-segment display and Vaman-Pygmy board are
listed in \autoref{tab:esp32-fpga-sevenseg-display-connections}.

\subsubsection{Building}
\begin{enumerate}
    \item Build the FPGA project .bin file by entering the following commands at
    a terminal window.
    \begin{lstlisting}
# The following variable can be sourced from .shellrc or .venv/bin/activate
export QORC_SDK_PATH=/path/to/pygmy-sdk
cd inter-chip/m4-fpga/sevenseg/codes/fpga
make
    \end{lstlisting}
    \item On building successfully, the .bin file is generated at
    \begin{lstlisting}
inter-chip/m4-fpga/sevenseg/codes/fpga/rtl/AL4S3B_FPGA_Top.bin
    \end{lstlisting}
    \item Edit the variable PROJ\_ROOT to point to the location of the pygmy-sdk
    folder on your system in the following file.
    \begin{lstlisting}
inter-chip/m4-fpga/sevenseg/codes/m4/GCC_Project/config.mk
    \end{lstlisting}
    \item Build the M4 project .bin file by entering the following commands at a
    terminal window.
    \begin{lstlisting}
cd inter-chip/m4-fpga/sevenseg/codes/m4/GCC_Project
make -j4
    \end{lstlisting}
    \item On building successfully, the .bin file is generated at
    \begin{lstlisting}
inter-chip/m4-fpga/sevenseg/codes/m4/GCC_Project/output/bin/m4.bin
    \end{lstlisting}
    \item Flash both FPGA and M4 .bin files to the Vaman-Pygmy by resetting it
    to bootloader mode and entering the following command at a terminal window.
    \begin{lstlisting}
python3 /path/to/tinyfpga-programmer-gui.py --port /dev/ttyACMxx --mode m4-fpga --m4app /path/to/m4.bin --appfpga /path/to/AL4S3B_FPGA_Top.bin --reset
    \end{lstlisting}
    where /dev/ttyACMxx is the port at which the Vaman board is available. This
    can be obtained by inspecting the output of the following command (requires
    root/sudo privileges).
    \begin{lstlisting}
dmesg -w
    \end{lstlisting}
\end{enumerate}

\subsubsection{Demonstration}
All three LEDs onboard the Vaman will toggle every one second when the Vaman
boots normally.

\subsubsection{Exercises}
\begin{enumerate}[resume]
    \item Create a decade counter with this interface.
    \item Extend the code to accommodate hexadecimal digits.
\end{enumerate}