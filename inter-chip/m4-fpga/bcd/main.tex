\subsection{Binary-Coded Decimal Decoder}
This subsection demonstrates the interface between the M4 subsystem and eFPGA
subsystem on the Vaman-Pygmy by simulating a Binary-Coded Decimal (BCD) Decoder,
otherwise known as a 7447 IC. The GPIO interface is exposed to the M4 by the
FPGA platform.

\subsubsection{Components}
The components required are listed in \autoref{tab:m4-fpga-bcd-components}.
\begin{table}[!ht]
    \centering
    \input{inter-chip/m4-fpga/bcd/tables/components.tex}
    \caption{Components Required for Simulating the 7447 IC.}
    \label{tab:m4-fpga-bcd-components}
\end{table}

\subsubsection{Connections}
The connections to be made between the seven-segment display and Vaman board are
listed in \autoref{tab:esp32-fpga-sevenseg-display-connections}. The equivalent
7447 IC input pins are listed in \autoref{tab:m4-fpga-bcd-bcd-connections}.

\begin{table}[!ht]
    \centering
    \input{inter-chip/m4-fpga/bcd/tables/bcd-connections.tex}
    \caption{Equivalent Input Pins for the 7447 IC on the Vaman-Pygmy.}
    \label{tab:m4-fpga-bcd-bcd-connections}
\end{table}

\subsubsection{Building}
\begin{enumerate}
    \item Build the FPGA project .bin file by entering the following commands at
    a terminal window.
    \begin{lstlisting}
# The following variable can be sourced from .shellrc or .venv/bin/activate
export QORC_SDK_PATH=/path/to/pygmy-sdk
cd inter-chip/m4-fpga/bcd/codes/fpga
make
    \end{lstlisting}
    \item On building successfully, the .bin file is generated at
    \begin{lstlisting}
inter-chip/m4-fpga/bcd/codes/fpga/rtl/AL4S3B_FPGA_Top.bin
    \end{lstlisting}
    \item Edit the variable PROJ\_ROOT to point to the location of the pygmy-sdk
    folder on your system in the file
    \begin{lstlisting}
inter-chip/m4-fpga/bcd/codes/m4/GCC_Project/config.mk
    \end{lstlisting}
    \item Build the M4 project .bin file by entering the following commands at a
    terminal window.
    \begin{lstlisting}
cd inter-chip/m4-fpga/bcd/codes/m4/GCC_Project
make -j4
    \end{lstlisting}
    \item On building successfully, the .bin file is generated at
    \begin{lstlisting}
inter-chip/m4-fpga/bcd/codes/m4/GCC_Project/output/bin/m4.bin
    \end{lstlisting}
    \item Flash both FPGA and M4 .bin files to the Vaman-Pygmy by resetting it
    to bootloader mode and entering the following command at a terminal window
    \begin{lstlisting}
python3 /path/to/tinyfpga-programmer-gui.py --port /dev/ttyACMxx --mode m4-fpga --m4app /path/to/m4.bin --appfpga /path/to/AL4S3B_FPGA_Top.bin --reset
    \end{lstlisting}
    where /dev/ttyACMxx is the port at which the Vaman board is available. This
    can be obtained by inspecting the output of the following command (requires
    root/sudo privileges).
    \begin{lstlisting}
dmesg -w
    \end{lstlisting}
\end{enumerate}

\subsubsection{Demonstration}
Give an input to the BCD by connecting the male ends of the jumpers at pins 10
to 13 on the Vaman-Pygmy to high voltage or ground. The equivalent digit should
be displayed on the seven-segment display.

\subsubsection{Exercises}
\begin{enumerate}[resume]
    \item Create code for an incrementing encoder instead of a display decoder.
    \item Extend the code to perform mathematical operations modulo 10 and
    display the result.
    \item Extend the code to accommodate hexadecimal digits.
\end{enumerate}